\newpage
\section{Contratti a oggetto informatico}
\subsection{Che cos'è un contratto?}
Il contratto è l'accordo di due o più parti per costituire, regolare o estinguere
tra loro un rapporto giuridico patrimoniale
\subsection{Il principio di autonomia contrattuale}
Le parti possono liberamente determinare il contenuto del contratto nei limiti di legge.
\subsection{Quali sono i requisiti di un contratto?}
Secondo l'Art. 1325 c.c. i requisiti del contratto sono:
\begin{itemize}
    \item l'accordo delle parti
    \item la causa
    \item l'oggetto
    \item la forma, quando è prescritta dalla legge a pena di nullità
\end{itemize}
[Vedi {Concetti giuridici di base - Contratto}]

\subsection{Quando è concluso un contratto?}
Il contratto è concluso nel momento in cui chi ha fatto la proposta ha conoscenza
dell'accettazione dell'altra parte. Questa accettazione deve giungere nel termine stabilito dal
proponente o in quello generalmente necessario a seconda della natura dell'affare o degli usi.\newline
Il proponente può ritenere efficace l'accettazione tardiva, purchè ne dia immediatamente avviso all'altra
parte. Se il  proponente richiede per l'accettazione una forma determinata, l'accettazione non ha
effetto se è data in forma diversa. Un'accettazione non conforme alla proposta equivale a nuova proposta. \newline
\textbf{Esecuzione prima della risposta dell'accettante}\newline
Qualora, su richiesta del proponente o per la natura dell'affare o secondo gli usi, la prestazione debba
eseguirsi senza una preventiva risposta, il contratto è concluso nel tempo e nel luogo in cui ha avuto
inizio l'esecuzione (es. pagamento del prezzo, consegna del bene). L'accettante deve dare prontamente avviso
all'altra parte dell'iniziata esecuzione e, in mancanza, è tenuto al risarcimento del danno. \newline
\textbf{Revoca della proposta e dell'accettazione}\newline
La proposta può essere revocata finchè il contratto non sia concluso. Tuttavia, se l'accettante
ne ha intrapreso in buona fede l'esecuzione prima di avere notizia della revoca, il proponente è tenuto a
indennizzarlo delle spese e delle perdite subite per l'iniziata esecuzione del contratto.\newline
L'accettazione può essere revocata, purchè la revoca giunga a conoscenza del proponente prima dell'accettazione.

\subsection{Cosa sono le condizioni generali di contratto?}
Le condizioni generali di contratto sono predisposte da uno dei contraenti. Sono efficaci nei confronti dell'altro, se
alla conclusione del contratto questi le ha conosciute o avrebbe dovuto conoscerle con ordinaria diligenza.
In ogni caso non hanno effetto, se non sono specificamente approvate per iscritto, le condizioni che stabiliscono,
a favore di colui che le ha predisposte:
\begin{itemize}
    \item limitazioni di responsabilità
    \item facoltà di recedere dal contratto
    \item facoltà di sospendere l'esecuzione
\end{itemize}
Sanciscono a carico dell'altro contraente:
\begin{itemize}
    \item decadenze
    \item limitazioni alla libertà contrattuale nei rapporti con terzi
    \item tacita proroga o rinnovazione del contratto
    \item clausole compromissorie
    \item deroghe alla competenza dell'autorità giudiziaria
\end{itemize}
Esempio [Art. 1229 c.c.]: sono nulle le clausole di esonero da responsabilità
del debitore per dolo o colpa grave o violazione di norme di ordine pubblico.

\subsection{Quali clausole delle condizioni generali di contratto devono essere specificamente approvate per iscritto?}
Le clausole vessatorie elencate nell'articolo 1341 comma 2 sono efficaci solo se specificamente
approvate per iscritto. Le clausole vessatorie sono clausole che, malgrado la buona fede,
determinano a carico del consumatore un significativo squilibrio dei diritti e degli obblighi derivanti
dal contratto.\newline
Definiamo \emph{consumatore} una persona fisica che agisce per scopi estranei all'attività imprenditoriale, commerciale,
artigianale o professionale eventualmente svolta.\newline
Un esempio di clausola vessatoria è: \newline \newline

consentire al professionista di trattenere una somma di denaro versata dal consumatore se quest'ultimo non conclude
il contratto o recede da esso, \emph{senza} prevedere il diritto del consumatore di esigere dal professionista
il doppio della somma corrisposta se è quest ultimo a non concludere il contratto oppure a recedere \newline \newline

Al fine di costituire una più forte tutela per il contraente debole, il legislatore, nell'ambito
delle condizioni generali di contratto, ha previsto la necessità della specifica approvazione per iscritto delle clausole
vessatorie, ossia di quelle pattuizione particolarmente onerose e svantaggiose per l'aderente. Questa sottoscrizione
è dunque doppia; è necessario infatti accettare le condizioni generali in blocco e le clausole vessatorie specificamente.

\subsection{Accertamento delle clausole vessatorie}
La vessatorietà di una clausola è valutata tenendo conto del bene o del servizio oggetto del contratto e facendo riferimento
alle circostanze esistenti al momento della sua conclusione e alle altre clausole del contratto. Questa valutazione non riguarda
la determinazione dell'oggetto del contratto, nè l'adeguatezza del corrispettivo dei beni e dei servizi, purchè siano individuati
in modo chiaro e comprensibile.\newline
Non sono vessatorie le clausole che riproducono disposizioni di legge contenute in convenzioni internazionali dell'Unione Europea.\newline
Non sono vessatorie le clausole o gli elementi di clausola che siano stati oggetto di trattativa individuale.\newline
Si ricorda che in caso di dubbio sul senso di una clausola, prevale l'interpretazione più favorevole al consumatore.\newline\newline
Si ricorda che le clausole vessatorie sono considerate nulle mentre il contratto rimane valido per il resto e sono nulle quelle clausole
che abbiano per oggetto o per effetto di, ad esempio, escludere o limitare la responsabilità del professionista in caso di
morte o danno al consumatore, risultante da un fatto o da un'omissione del professionista

\subsection{Come si può gestire la doppia sottoscrizione nei contratti on line?}
\begin{itemize}
    \item Predisposizione di un apposito form con lo specifico richiamo di
        tutte le clausole vessatorie del regolamento, per la cui accettazione è
        richiesta un'apposita pressione del “tasto virtuale”, magari
        accompagnata da un'ulteriore login con digitazione di username e
        password dell'aderente, che valga da firma elettronica (semplice)
    \item Prevedere l'obbligo che le clausole vessatorie appaiano sul video del
        contraente in forma particolarmente chiara, attraverso
        un'evidenziazione delle stesse o l'uso di caratteri o colori diversi,
        magari imponendo al contraente di “scorrere” tutta la pagina in modo
        tale da essere forzato a leggere quanto così evidenziato
    \item Prevedere due specifici, separati ed autonomi campi per le condizioni
        generali di contratto e per le clausole vessatorie, così che la
        conclusione del contratto non si verifichi nel caso in cui il Cliente
        clicchi sul campo “SI” delle sole condizioni generali, senza poi
        accettare specificamente – cliccando su di un altro, apposito e
        distinto campo, spesso all'interno della stessa pagina Web – le
        clausole vessatorie
    \item Inviare al cliente copia del contratto scritto con tutte le
        condizioni approvate via web e ottenere da questi una copia
        controfirmata
\end{itemize}

\subsection{Contratto concluso mediante moduli o formulari}
Nei contratti conclusi mediante la sottoscrizione di moduli o formulari, predisposti per disciplinare in maniera uniforme determinati
rapporti contrattuali, le clausole aggiunte al modulo o al formulario prevalgono su quelle del modulo o del formulario qualora siano
incompatibili con esse, anche se queste ultime non sono state cancellate

\subsection{Contratto di licenza d'uso software}
Il contratto di licenza d'uso software è un \textbf{contratto atipico} in cui il licenziante cede al licenziatario il diritto di godimento del software e la
documentazione accessoria per il tempo stabilito. Se la licenza è gratuita si parla di \textbf{freeware}. \newline
In questa licenza \textbf{non si cede} la titolarità del programma e non si cede il diritto di sfruttamento economico.\newline
Possono essere applicate le norme del codice civile sulla locazione per quanto compatibili e salvo accordo contrario.\newline
Il licenziante è il titolare dei diritti di utilizzazione economica e mantiene il diritto esclusivo di riprodurre il
programma in maniera permanente o temporanea, totale o parziale; di modificarlo, tradurlo, adattarlo e trasformarlo; di
distribuirlo in qualsiasi forma (che equivale ai diritti di utilizzazione economica).\newline
Il licenziatario è colui che acquista il diritto di utilizzare il programma nei limiti previsti dalla legge sul
diritto d'autore.

\subsection{Garanzie e responsabilità per vizi}
Se il software non permette di ottenere alcun risultato utile a causa dei vizi, l'utente può ottenere: restituzione del prezzo
o riparazione o sostituzione software difettoso o pagamento di una somma. Le clausole in genere
escludono responsabilità per danni dovuti all'interruzione dell'attività; cioè le clausole che
escludono responsabilità extra-contrattuale sono valide se riguardano un danno patrimoniale, mentre sono
nulle se riguardano danno a persone.

\subsection{Licenza a strappo}
Riguarda software destinato ad ampia diffusione, mediante distribuzione di massa. In questo caso c'è la necessità
di uniformare e velocizzare la commercializzazione.\newline
Il software è confezionato in un involucro trasparente su cui sono leggibili le condizioni di contratto. La clausola
principale prevede che l'apertura della confezione sigillata comporti l'accettazione del contratto.\newline
In questo tipo di contratti le clausole vessatorie potrebbero essere ritenute inefficaci poichè
diviene impossibile approvarle specificamente

\subsection{Contratto di sviluppo software}
Una parte (es. software house o professionista informatico) si obbliga a studiare, sviluppare e realizzare un software in base alle richieste dell'altra parte (committente).\newline
Esiste una diversa qualificazione giuridica a seconda del soggetto incaricato:
\begin{itemize}
    \item imprenditore: contratto di appalto di servizi. Cioè un contratto con cui una parte (appaltatore) assume con organizzazione dei mezzi
    necessari e con gestione a proprio rischio il compimento di un'opera o di un servizio per conto di un appaltante (committente) verso un corrispettivo
    in denaro. In questo caso vi è obbligo di risultato e vige il divieto di subappalto senza autorizzazione del committente.\newline
    La proprietà dell'opera si trasferisce al committente con la consegna.\newline
    Riguardo l'appalto, nel codice civile si trovano le principali norme applicabili allo sviluppo software:
    \begin{itemize}
        \item L'appaltatore non può apportare variazioni alle modalità convenute dell'opera se il committente non le ha autorizzate
        \item Potere di verifica del committente per valutare lo stato di avanzamento dello sviluppo del software
        \item Il committente, prima di ricevere la consegna, ha diritto di verificare l'opera compiuta
        \item L'appaltatore è tenuto alla garanzia per le difformità e i vizi dell'opera. La garanzia non è dovuta se il committente ha
        accettato l'opera e le difformità o i vizi erano da lui conosciuti o riconoscibili. Il committente deve denunziare all'appaltatore le
        difformità o i vizi entro 60 giorni dalla scoperta.
        \item Il committente può chiedere che le difformità o i vizi siano eliminati a spese dell'appaltatore, oppure che il prezzo
        sia proporzionalmente diminuito. Se le difformità rendono del tutto inadatta l'opera alla sua destinazione, il committente può chiedere la
        risoluzione del contratto.
    \end{itemize}
    \item professionista: contratto di prestazione d'opera intellettuale. Il professionista si avvale delle proprie competenze professionali,
    senza organizzazione di impresa, e lavora senza vincolo di subordinazione verso il committente.\newline
    Si dice che si ha un'obbligazione di mezzi, non di risultato. Il che non vuol dire non essere tenuti a fornire il risultato promesso. Per obbligazione
    di mezzi si intende fare tutto quanto è nelle proprie possibilità per soddisfare l'interesse del creditore, senza garantire la realizzazione
    del risultato (si pensi ad un avvocato nella vittoria di una causa, o ad un medico nei confronti di un paziente che è in pericolo di vita).\newline
    Nelle ipotesi di obbligazioni di mezzi, potrà intentarsi causa nei confronti del debitore in caso di mancato svolgimento dell'attività nel modo
    dovuto, cioè secondo diligenza.
\end{itemize}

\subsection{Analisi attenta di un contratto}
\subsubsection{Contratto d'opera e contratto d'opera intellettuale}
Quando una persona si obbliga a compiere verso un corrispettivo un'opera o un servizio, con lavoro prevalentemente proprio e senza vincolo di subordinazione
nei confronti del committente, si parla di contratto d'opera. \newline
Il contratto d'opera è intellettuale se ha per oggetto un'opera intellettuale.
\subsubsection{Oggetto del contratto}
L'oggetto del contratto deve essere possibile, lecito, determinato o determinabile.\newline
Bisogna descrivere chiaramente e in modo non vago l'oggetto del contratto:
\begin{itemize}
    \item Quale software dovrà essere sviluppato (obiettivi, caratteristiche, eventualmente in un allegato tecnico)
    \item In quale modo il software sarà rilasciato (es. installato presso il client, SaaS)
    \item A che titolo se ne prevede la realizzazione e consegna (es. l'ingegnere realizza il software e ne cede i diritti di
    utilizzazione economica)
\end{itemize}
\subsubsection{Esecuzione dell'opera}
Se il prestatore d'opera non procede all'esecuzione dell'opera secondo le condizioni stabilite dal contratto e a regola d'arte,
il committente può fissare un congruo termine, entro il quale il prestatore d'opera deve conformarsi a tali condizioni. Trascorso il termine fissato, il
committente può recedere dal contratto, salvo il diritto al risarcimento dei danni.\newline
Nei documenti del contratto occorre sempre richiamare gli allegati e specificare che costituiscono parte integrante del contratto
stesso.\newline
Molto importante è definire dettagliatamente i criteri di verifica delle varie fasi del software. Sarà in base ad essi
che si determinerà l'eventuale inadempimento dell'ingegnere rispetto agli obblighi assunti.

\subsubsection{Obblighi del committente}
Tra gli obblighi del committente troviamo l'indicare le informazioni, la documentazione e/o i materiali che il committente deve
eventualmente fornire all'ingegnere per la realizzazione del software.\newline
Indicare i termini entro i quali deve avvenire la consegna da parte del committente.\newline
Può essere utile prevedere una clausola con la quale il committente si impegna a garantire e a tenere indenne l'ingegnere da qualsiasi reclamo,
causa, danno o spesa dipendente dalla carenza, mancanza di qualità o pretese di terzi relativi alle informazioni,
alla documentazione e ai materiali forniti dal committente.

\subsubsection{Garanzie e responsabilità}
L'accettazione espressa o tacita dell'opera libera il prestatore d'opera dalla
responsabilità per difformità o per vizi della medesima, se all'atto
dell'accettazione questi erano noti al committente o facilmente riconoscibili,
purchè in questo caso non siano stati dolosamente occultati.\newline
Il committente deve, a pena di decadenza, denunziare le difformità e i vizi occulti al prestatore d'opera entro otto giorni
dalla scoperta. L'azione si prescrive entro un anno dalla consegna.\newline
Nell'adempimento delle obbligazioni inerenti all'esercizio di un'attività
professionale, la diligenza deve valutarsi con riguardo alla natura
dell'attività esercitata.\newline
Se la prestazione implica la soluzione di problemi tecnici di speciale
difficoltà, il prestatore d'opera non risponde dei danni,
se non in caso di dolo o di colpa grave.\newline
È buona pratica indicare sempre quali garanzie si forniscono in relazione al Software e per quale periodo le garanzie sussistono (es.
buona progettazione, assenza di file nocivi ecc.); inoltre va indicato cosa si garantisce al committente in caso di vizi o difetti
(es. correzione, ricerca di una soluzione alternativa ecc.).\newline
Una volta che il committente ha accettato i risultati del software, l'ingegnere è liberato da ogni responsabilità.

\subsubsection{Responsabilità per danni}
Indicare entro quali limiti si risponde di eventuali danni causati al committente o a terzi. Ad esempio limitazione ai danni
diretti effettivamente causati, derivanti da atti od omissioni dell'ingegnere nell'esecuzione del contratto, fino ad un massimo
di TOT euro

\subsubsection{Variazioni richieste}
Specificare se ed entro quali limiti il committente può apportare delle variazioni alle specifiche tecniche del software. Indicare inoltre la modalità
con la quale deve avvenire la richiesta di variazioni (es. PEC, raccomandata a/r, fax, ecc.) e i tempi entro i quali
le variazioni non necessarie possano essere richieste.

\subsubsection{Variazioni necessarie}
Se l'esecuzione dell'opera diventa impossibile per causa non imputabile ad alcuna delle parti, il prestatore d'opera ha diritto ad un compenso per il lavoro
prestato in relazione all'utilità della parte dell'opera compiuta.\newline
Specificare cosa si dovrà fare in caso di eventi sopravvenuti imprevedibili e non imputabili ad alcuna delle parti che rendano necessario apportare delle variazioni
alle specifiche tecniche del software e alle tempistiche di realizzazione.\newline
Specificare quali sono gli impegni delle parti per concordare le variazioni da introdurre e il correlativo adeguamento
delle tempistiche di realizzazione e di prezzo.
È utile prevedere la possibilità di recesso dell'ingegnere e/o del committente quando le
variazioni risultino troppo onerose e/o comportino l'impossibilità nella prosecuzione della
realizzazione del software.\newline
In caso di recesso, stabilire i criteri per la quantificazione dell'equa indennità spettante all'ingegnere in ragione dell'attività
fino a quel momento prestata.

\subsubsection{Diritti di proprietà intellettuale}
Concordare quale delle parti diverrà titolare dei diritti di utilizzazione economica del software. Fatto questo specificare
se ed eventualmente quale uso potrà essere posto in essere dalla parte non titolare del software (es. l'ingegnere potrebbe
voler riutilizzare determinate componenti del software per lavori futuri). Specificare prima dell'inizio dell'esecuzione del contratto
a quale titolo possono essere utilizzati determinati componenti.

\subsubsection{Recesso delle parti}
Disciplinare il recesso delle parti considerando che in caso di prestazione d'opera il committente può
recedere dal contratto, rimborsando all'ingegnere le spese sostenute, pagando il compenso per l'opera svolta
e pagando un indennizzo per il mancato guadagno.

\subsubsection{Risoluzione del contratto}
I contraenti possono convenire espressamente che il contratto si risolva nel caso che una determinata obbligazione non sia adempiuta secondo le modalità stabilite.
In questo caso, la risoluzione si verifica di diritto quando la parte interessata dichiara all'altra che intende valersi della clausola risolutiva.
Se il termine fissato per la prestazione di una delle parti deve considerarsi essenziale nell'interesse dell'altra, questa, salvo patto o uso contrario,
se vuole esigerne l'esecuzione nonostante la scadenza del termine, deve darne notizia all'altra parte entro tre giorni. In mancanza, il contratto s'intende
risoluto di diritto anche se non è stata espressamente pattuita la risoluzione.\newline
La risoluzione del contratto per inadempimento ha effetto retroattivo tra le parti, salvo il caso di contratti ad esecuzione continuata o periodica,
riguardo ai quali l'effetto della risoluzione non si estende alle prestazioni già eseguite.\newline
Specificare quali sono gli eventi che le parti vogliono considerare come cause di risoluzione del contratto (es. mancato pagamento o ritardo nel pagamento oltre un certo limite)

\subsection{A quali clausole della licenza bisogna prestare attenzione se si vuole utilizzare una libreria software?}
Bisogna prestare particolare attenzione alle clausole più importanti per la tutela dell'ingegnere, ad esempio la clausola sulla riservatezza. In questo caso se
l'ingegnere non può neppure fare menzione del fatto che ha sviluppato un software per un determinato committente, questo potrà costituire una
limitazione per il suo curriculum.

\subsection{Licenza di software libero / Open Source e licenze Creative Commons Italia}
Si tratta di una particolare licenza software che prevede la messa a disposizione del codice sorgente. Essa \textbf{può} prevedere
l'autorizzazione a usare, modificare, integrare, riprodurre, distribuire il programma. \newline
Si tratta di un negozio giuridico con cui il titolare dei diritti di utilizzazione economica concede - in genere -
a titolo gratuito determinate facoltà che gli spettano in base alla legge sul diritto d'autore. \newline\newline
Nelle Licenze Creative Commons Italia l'autore decide autonomamente a quali condizioni condividere l'opera, concedendo
o meno tutti o alcuni diritti degli altri utenti.\newline
In base a quanto stabilito nella licenza prescelta, ad altri utenti sarà concesso o proibito di modificare l'opera, di utilizzarla o di includerla
in altre opere.\newline
Si riportano le condizioni previste dalle licenze Creative Commons Italia:
\begin{itemize}
    \item \textbf{Attribuzione}: la licenza "Attribuzione 3.0 Italia" consente di riprodurre, distribuire, comunicare al pubblico,
    esporre in pubblico, rappresentare, eseguire, modificare, usare per fini commerciali l'opera, a condizione che sia
    attribuita la paternità dell'opera nei modi indicati dall'autore o da chi ha concesso l'opera in licenza e in modo tale da non suggerire che essi avallino l'utente
    o il modo in cui questi usa l'opera
    \item \textbf{Attribuzione - Non opere derivate} la licenza "Attribuzione - Non opere derivate 3.0 Italia" consente l'Attribuzione ma l'utente non
    può alterare o trasformare l'opera, nè usarla pre crearne un'altra.
    \item \textbf{Attribuzione - Non commerciale} la licenza "Attribuzione - Non commerciale 3.0 Italia" consente l'Attribuzione ma l'opera non può essere utilizzata
    per fini commerciali
    \item \textbf{Attribuzione - Non commerciale - Non opere derivate} la licenza "Attribuzione - Non commerciale - Non opere derivate 3.0 Italia" consente l'Attribuzione ma l'opera non
    può essere utilizzta per fini commerciali e l'utente non può alterare o trasformare l'opera, nè usarla pre crearne un'altra.
    \item \textbf{Attribuzione - Non commerciale - Condividi allo stesso modo} la licenza "Attribuzione - Non commerciale - Condividi allo stesso modo 3.0 Italia" consente l'Attribuzione ma
    l'opera non può essere utilizzata per fini commerciali e se l'utente altera o trasforma l'opera, o se la usa per crearne un'altra, può distribuire l'opera
    che ne deriva solo con una licenza identica o equivalente a questa
    \item \textbf{Attribuzione - Condividi allo stesso modo} la licenza "Attribuzione - Condividi allo stesso modo" consente l'Attribuzione e se l'utente altera o trasforma l'opera, o se la
    usa per crearne un'altra, può distribuire l'opera che ne deriva solo con una licenza identica o equivalente a questa.
    \item \textbf{Donazione al pubblico dominio} la licenza CC0 1.0 Universal (CC0 1.0) prevede che l'opera sia dedicata al pubblico dominio. \newline
    L'autore rinuncia a tutti i suoi diritti sull'opera in tutto il mondo come previsti dalla leggi sul diritto d'autore, inclusi tutti i diritti connessi
    al diritto d'autore o affini, nella misura consentita dalla legge.\newline
    Gli utenti possono copiare, modificare, distribuire ed utilizzare l'opera, anche per fini commerciali, senza chiedere
    alcun permesso. (es. www.pixabay.com)
    \item \textbf{CC Plus} le licenze CC sono non esclusive, è sempre possibile aggiungere accordi che - senza ridurre i diritti conferiti dalla licenza - offrano, a certe
    condizioni, possiblità aggiuntive a tutti o ad alcuni licenziatari. Questo è il protocollo seguito, per esempio, nell'ambito del "protocollo" CCPlus.
\end{itemize}
La licenza CC Italia non ha effetto in nessun modo su diritti come eccezioni,
libere utilizzazioni, diritti morali d'autore, altri diritti che persone
possono avere sia sull'opera stessa
che su come viene utilizzata (es. diritto all'immagine o alla tutela dei dati personali)


\newpage
\section{Ricerca e analisi delle fonti del diritto}

\subsection{Come procederebbe per risolvere un problema che richiede l'applicazione di norme giuridiche?}

Per risolvere un problema che richiede l'applicazione di norme giuridiche bisogna \emph{cercare} il diritto pertinente al caso e \emph{applicarlo}.\newline
Questo processo è possibile scomporlo, dunque, in:
\begin{itemize}
    \item \textbf{Reperimento del diritto} da effettuare con scrupolosità e consapevolezza data la grande mole di informazioni a cui
                                           la rete ci espone.\newline
        Si cercano:
        \begin{itemize}
            \item Atti normativi presenti nell'ordinamento giuridico (es. direttive, leggi, decreti, regolamenti, ecc.).\newline
                  Gli atti normativi possono essere emanati da autorità di diversi livelli (Internazionale, Comunitario, Nazionale, Locale)
                  e vengono esposti in elenchi che rimandano (1-1) alle raccolte. Alcune normative possono essere pubblicate da privati (es.
                  De Agostini). \newline
                  Ogni atto ha una natura, un numero progressivo, una data e un titolo.
            \item Ordinamento della giurisprudenza attraverso le decisioni dei giudici di ogni ordine e grado e delle interpretazioni
                  in esse contenute
            \item Interpretazioni dottrinali: principi fondamentali, teorie, nozioni elaborate da studiosi giurisprudenziali
        \end{itemize}
    \item Sviluppo di un ragionamento per arrivare a una soluzione
\end{itemize}

\subsection{Che cosa è opportuno cercare quando si effettua una ricerca giuridica?}
Quando si effettua una ricerca giuridica è opportuno cercare normative, giurisprudenza e dottrina, fonti terziarie
 su Internet o banche giuridiche. \newline
Occorre prestare attenzione alla completezza delle norme, se si tratta della versione in vigore attualmente,
alle diverse interpretazioni giurisprudenziali ecc. \newline
\newline
\textbf{Può risultare molto utile cercare giurisprudenza e dottrina}. Infatti:
\begin{itemize}
    \item \textbf{Giurisprudenza}: decisioni prese da soggetti istituzionali in base a criteri di competenza (territoriale, per materia, per valore). \newline
            In questo caso la pronuncia giudiziale (es. \textbf{sentenza}) può essere reperita come \textbf{massima} o come \textbf{versione per esteso}.
            La prima è un breve riassunto dei principi fondamentali che possono essere tratti dalla sentenza. La seconda equivale al testo integrale
    \item \textbf{Dottrina}: si tratta delle pubblicazioni scritte dagli studiosi del diritto. Queste servono da supporto all'interpretazione e alla conoscenza
            della normativa o della giurisprudenza (es. rassegne, articoli su riviste giuridiche ecc.)
\end{itemize}

\subsection{Differenza tra testo storico e testo consolidato di una norma}
L'unica versione delle leggi, dei decreti, dei regolamenti che fa fede a tutti gli effetti di legge è quella pubblicata in Gazzetta Ufficiale.\newline
Il testo di una norma pubblicato per la prima volta in Gazzetta Ufficiale è il \textbf{testo storico}.\newline
Il testo storico può essere successivamente modificato da altre norme che non modificano la norma pubblicata in
Gazzetta Ufficiale. Su quest' ultima si troveranno solo i riferimenti alle modifiche.\newline
\textbf{Il testo consolidato} è l'effettivo testo in vigore e può essere reperito nelle banche dati giuridiche.\newline
Di fatto per capire se una legge trovata sul web è quella più aggiornata bisogna controllare gli ultimi riferimenti alle modifiche della GU.

\subsection{Autorità Garanti}
Una autorità garante è un organo creato per sorvegliare lo svolgimento di attività economiche realizzate in regime di monopolio o caratterizzate da uno speciale
interesse generale. Ha in genere lo scopo di salvaguardare cittadini e imprese da situazioni che li vede in posizione di debolezza rispetto agli operatori che producono
ed erogano i beni o i servizi in regime di monopolio, quasi-monopolio, oligopolio, ovvero pure in quelle situazioni in cui esistono forti asimmetrie informative. Può
essere dotata di capacità sanzionatoria. \newline
Le Autorità Garanti propagano delle \textbf{pronunce} che contribuiscono a garantire l'esatta osservanza e l'interpretazione uniforme della legge. Queste pronunce svolgono
una funzione para-giurisprudenziale. La loro qualificazione giuridica è infatti dibattuta.

\subsection{Come faccio a sapere che una legge (per es. trovata sul web) è in vigore e non è una vecchia
versione abrogata?}

Occorre consultare il sito ufficiale dello stat italiano \textit{normattiva}, che è una raccolta, curata dall'istituto poligrafico dello Stato, che indica la data di entrata in vigore dei provvedimenti e se sono vigenti o meno.

\subsection{Gli atti ufficiali dello stato sono coperti da diritto d'autore?}
No.
\newpage
\section{I reati informatici}

\subsection{Che cosa sono i reati informatici?}
Un reato informatico è un illecito penale, punibile con pene pecuniarie o
detentive, che viene commesso mediante tecnologie informatiche o a danno di
tecnologie informatiche.

\subsection{Posso applicare per analogia la legge penale? Perché?}
No, perchè nessuno può essere punito per un fatto che in quel momento non è
considerato reato dalla legge.

\subsection{Quali sono i reati informatici di maggiore interesse per l’ingegnere?}
\begin{itemize}
    \item Salvo che il fatto costituisca più grave reato, chiunque distrugge,
        deteriora, cancella, altera o sopprime informazioni, dati o programmi
        informatici altrui è punito, a querela della persona offesa, con la
        reclusione da sei mesi a tre anni. Se il fatto è commesso con violenza
        alla persona o con minaccia ovvero con abuso della qualità di operatore
        del sistema, la pena è della reclusione da uno a quattro anni.
        \newline È particolarmente grave il caso in cui un fatto del genere
        riguardi informazioni, dati o programmi informatici dello Stato, di un
        ente pubblico o ad essi pertinenti. In questo caso la reclusione va da
        uno a quattro anni.
    \item Chiunque, fuori dai casi consentiti dalla legge, installa
        apparecchiature atte ad intercettare,impedire o interrompere
        comunicazioni relative ad un sistema informatico o telematico ovvero
        intercorrenti tra più sistemi, è punito con la reclusione da uno a
        quattro anni. 
    \item Chiunque prende cognizione del contenuto di una corrispondenza
        (effettuata con ogni forma di comunicazione a distanza) chiusa, a lui
        non diretta, ovvero sottrae o distrae, al fine di prenderne o di farne
        da altri prendere cognizione, una corrispondenza chiusa o aperta, a lui
        non diretta, ovvero, in tutto o in parte, la distrugge o sopprime, è
        punito se il fatto non è previsto come reato da altra disposizione di
        legge, con la reclusione fino a un anno o con la multa da euro 50 a
        euro 516
\end{itemize}

\subsection{Accesso abusivo ad un sistema informatico}
Chiunque abusivamente si introduce in un sistema informatico o telematico
protetto da misure di sicurezza ovvero vi si mantiene contro la volontà
espressa o tacita di chi ha il diritto di escluderlo, è punito con la
reclusione fino a tre anni.
Vi sono ulteriori aggravanti che prevedono una pena da uno a cinque anni di reclusione se:
\begin{itemize}
    \item il fatto è commesso da un pubblico ufficiale o da un incaricato di un
        pubblico servizio, con abuso dei potere o con violazione dei doveri
        inerenti alla funzione o al servizio, o da chi esercita anche
        abusivamente la professione di investigatore privato, o con abuso della
        qualità di operatore del sistema 
    \item il colpevole per commettere il fatto usa violenza sulle cose o alle
        persone, ovvero se è palesamente armato
    \item dal fatto deriva la distruzione o il danneggiamento del sistema o
        l'interruzione totale o parziale del suo funzionamento, ovvero la
        distruzione o il danneggiamento dei dati, delle informazioni o dei
        programmi in esso contenuti
\end{itemize}
Qualora i fatti riguardino sistemi informatici o telematici di interesse
militare o relativi all'ordine pubblico, alla sicurezza pubblica, alla sanità o
alla protezione civile o comunque di interesse pubblico, la pena di reclusione
è rispettivamente da uno a cinque anni e da tre a otto anni.
\newline
Nel caso previsto dal primo comma il delitto è punibile a querela della persona
offesa; negli altri casi si procede d'ufficio.

\subsection{Frode informatica}
Chiunque alteri in qualsiasi modo il funzionamento di un sistema informatico o
telematico o intervenga senza diritto con qualsiasi modalità su dati,
informazioni o programmi contenuti in un sistema informatico o telematico o ad
esso pertinenti, procurando a sé o ad altri un ingiusto profitto con altrui
danno, è punito con la reclusione da sei mesi a tre anni e con la multa da euro
51 a euro 1.032.
\newline
Le aggravanti per questo reato informatico sono:
\begin{enumerate}
    \item la pena è della reclusione da uno a cinque anni e della multa da 309
        a 1.549€ se il fatto è commesso a danno dello Stato o di un altro ente
        pubblico
    \item se il fatto è commesso con abuso della qualità di operatore del sistema
    \item la pena è della reclusione da due a sei anni e della multa da euro
        600 a euro 3.000 se il fatto è commesso con furto o indebito utilizzo
        dell'identità digitale in danno di uno o più soggetti
\end{enumerate}
Inoltre il delitto è punibile a querela della persona offesa, salvo che ricorra
taluna delle circostanze aggravanti (1,2,3) o un'altra circostanza aggravante.
